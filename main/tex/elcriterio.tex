\begin{propoTorsionNoRamificada}\label{propo:torsionnoramificada}
	Sea $E/\cal K$ una curva el\'{\i}ptica tal quje los grupos de $m$-%
	torsi\'{o}n $E[m]$ son no ramificados para una cantidad infinita de
	enteros $m\geq 1$ y coprimos con $\char k$. Entonces
	$E$ tiene buena reducci\'{o}n.
\end{propoTorsionNoRamificada}

\begin{proof}
	Se tienen dos sucesiones exactas cortas de grupos
	\begin{center}
		\begin{tikzcd}
			0\arrow[r] & E_0\arrow[r] & E\arrow[r] &
				E/E_0\arrow[r] & 0 \quad\text{y} \\
			0\arrow[r] & E_1\arrow[r] & E_0\arrow[r] &
				\redns{E}\arrow[r] & 0
		\end{tikzcd}
	\end{center}
	Si $\closnr{\cal K}$ es la extensi\'{o}n no ramificada maximal de
	$\cal K$ en $\clos{\cal K}$, el cuerpo residual de $\closnr{\cal K}$ es
	$\clos k$. Usando la de modelos de N\'{e}ron para una curva
	el\'{\i}ptica, ``esquemas en grupos con fibra geom\'{e}trica $E/k$'',
	se puede decir lo siguiente:
	\begin{itemize}
		\item[(i)] Si $E/\cal K$ tiene reducci\'{o}n split
			multiplicativa, entonces $E(\cal K)/E_0(\cal K)$ es
			c\'{\i}clico de orden $v(\discriminante)$;
		\item[(ii)] en cualquier otro caso, dicho cociente es un grupo
			finito de orden, a lo sumo, $4$.
	\end{itemize}
	%
	Algo de esto se hab\'{\i}a demostrado en clase, pero la
	demostraci\'{o}n se basaba en la finitud del cuerpo residual, que,
	ciertamente, no se cumple para $\closnr{\cal K}$. Bajo las
	hip\'{o}tesis de la proposici\'{o}n, se puede deducir que existe un
	(existen infinitos) enteros $m\geq 1$ que cumple(n) simult\'{a}neamente
	con:
	\begin{itemize}
		\item[(i)] $(m,\char k)=1$,
		\item[(ii)]
			\begin{math}
				m>\#\big(E(\closnr{\cal K})/
					E_0(\closnr{\cal K})\big)
			\end{math} y
		\item[(iii)] $E[m]$ es no ramificado.
	\end{itemize}
	%
	En particular, la $m$-torsi\'{o}n est\'{a} contenida en
	$E(\closnr{\cal K})$ y $E(\closnr{\cal K})$ contiene un subgrupo $A$
	isomorfo a $\big(\bb Z/m\bb Z\big)^2$. Ahora bien,
	\begin{align*}
		E_0(\closnr{\cal K})\,\cap\,A & \,\leq\, A\,\simeq\,
			\bb Z/m\bb Z\times\bb Z/m\bb Z
		\text{ ,}
	\end{align*}
	%
	lo que implica que
	\begin{align*}
		E_0(\closnr{\cal K})\,\cap\,A & \,\simeq\,
			\bb Z/a\bb Z\times\bb Z/b\bb Z
		\text{ ,}
	\end{align*}
	%
	con $a,b\mid m$. Adem\'{a}s,
	\begin{align*}
		E(\closnr{\cal K})/E_0(\closnr{\cal K}) & \,\supset\,
			E_0(\closnr{\cal K})\cdot A/E_0(\closnr{\cal K})
			\,\simeq\,A/E_0(\closnr{\cal K})\cap A
		\text{ .}
	\end{align*}
	%
	As\'{\i}, $m>\frac{m^2}{ab}$, lo quje fuerza que exista un primo $l$
	que divida tanto a $a$ como a $b$. En definitiva, existe un subgrupo
	$B$ de $E_0(\closnr{\cal K})$ isomorfo a $\big(\bb Z/l\bb Z\big)^2$.
	Usando
	\begin{center}
		\begin{tikzcd}
			0\arrow[r] & E_1(\closnr{\cal K})\arrow[r] &
				E_0(\closnr{\cal K})\arrow[r] &
				\redns{E}(\clos k)\arrow[r] & 0
			\text{ ,}
		\end{tikzcd}
	\end{center}
	que $l\not =\char k$ y que $E_1(\closnr{\cal K})$ no tiene puntos de
	orden $l$,
	\begin{align*}
		E_1(\closnr{\cal K})\,\cap\,B & \,=\,1
		\text{ ,}
	\end{align*}
	%
	y, entonces, este grupo pasa a la curva reducida, a sus puntos no
	singulares. Ahora, si $E$ tuviese mala reducci\'{o}n, habr\'{\i}a dos
	posibilidades:
	\begin{itemize}
		\item[(i)] $\redns{E}(\clos k)\simeq\clos k^\times$
			(red. mult.), o bien
		\item[(ii)] $\redns{E}(\clos k)\simeq (\clos k,+)$ (red. ad.).
	\end{itemize}
	%
	El segundo caso no es viable, porque no hay $l$-torsi\'{o}n en
	$(\clos k,+)$. Por otra parte, en $\clos k^\times$, la $l$-torsi\'{o}n
	consiste en las ra\'{\i}ces $l$-\'{e}simas de la unidad en $\clos k$, y
	\'{e}stas constituyen un grupo c\'{\i}clico de orden $l$. La primera de
	las opciones, tampoco es posible. En consecuencia, $E/\closnr{\cal K}$
	tiene buena reducci\'{o}n. Como bajo toda extensi\'{o}n no ramificada,
	el tipo de reducci\'{o}n se preserva, $E$ tampoco tiene mala
	reducci\'{o}n sobre $\cal K$.
\end{proof}

\begin{teoNOS}\label{thm:NOS}
	Las siguientes son equivalentes:
	\begin{itemize}
		\item[(i)] $E$ tiene buena reducci\'{o}n;
		\item[(ii)] $E[m]$ es no ramificado para todo $m\geq 1$ coprimo
			con $\char k$;
		\item[(iii)] $\tate[l](E)$ es no ramificado para (alg\'{u}n)
			primo $l$ coprimo con $\char k$;
		\item[(iv)] $E[m]$ es no ramificado para infinitos enteros
			$m\geq 1$ coprimos con $\char k$.
	\end{itemize}
	%
\end{teoNOS}

\'{E}ste es el \emph{criterio de N\'{e}ron-Ogg-Shafarevich}. Este ``criterio''
se puede ver desde el lado de las representaciones.

Sea $E/\bb Q$ una curva el\'{\i}ptica dada por ecuaci\'{o}n de Weierstrass,
\eqref{eq:formadeweierstrass}, con $a_i\in\bb Q$. Haciendo cambios de variable
de la forma
\begin{align*}
	(x,y) & \,=\,(u^2x',u^3y')
	\text{ ,}
\end{align*}
%
se obtiene una nueva ecuaci\'{o}n con coeficientes
\begin{align*}
	a_i' & \,=\, a_i/u^i
	\text{ .}
\end{align*}
%
(Se puede llevar a una ecuaci\'{o}n con coeficientes enteros). Sea $m_p(E)$ la
menor potencia de $p$ que divide al discriminante $\Delta$ de alguna de las
ecuaciones con coeficientes enteros equivalente a la ecuaci\'{o}n que define a
$E$ --dos ecuaciones se dicen \emph{equivalentes}, si est\'{a}n relacionadas
por un cambio de variables ``admisible''. Se define el \emph{discriminante %
global de $E$} como
\begin{align*}
	\mindisc(E) & \,:=\,\prod_p\,p^{m_p(E)}
\end{align*}
%
(si $p\nmid\discriminante(E)$, entonces $m_p(E):=0$). Es posible llevar la
ecuaci\'{o}n de $E$, v\'{\i}a cambios de variable admisibles, a una cuyo
discriminante minimice todas las valuaciones, es decir, existe una ecuaci\'{o}n
equivalente $E'$ tal que
\begin{math}
	m_p(E)=v_p(\discriminante(E'))
\end{math}. As\'{\i}, $\discriminante(E')=\mindisc(E)$. Una ecuaci\'{o}n $E'$
con estas caracter\'{\i}sticas se denomina \emph{ecuaci\'{o}n de Weierstrass %
minimal global}.

Sea $N$ el conductor de $E$, divisible exactamente por los primos de mala
reducci\'{o}n. Dado un primo $l$, el m\'{o}dulo de Tate $\tate[l](E)$
proporciona una representaci\'{o}n de
\begin{math}
	G=\galois{\clos{\bb Q}/\bb Q}
\end{math}, un morfismo continuo
\begin{align*}
	\rho & \,=\,\rho_{E,l} \,:\,G\,\rightarrow\,\GL[2](\bb Z_l)\,\subset\,
		\GL[2](\bb Q_l)
	\text{ .}
\end{align*}
%
Sea $p\in\bb Z$ un primo arbitrario y sea $\frak p\subset\clos{\bb Z}$ primo
maximal arriba de $p$ ($\frak p=\ker(\clos{\bb Z}\rightarrow\clos{\bb F_p})$).
Se definen
\begin{align*}
	\descomp[\frak p] & \,:=\,\big\{\sigma\in G\,:\,
		\frak p^\sigma=\frak p\big\} \quad\text{e} \\
	\inercia[\frak p] & \,:=\,\big\{\sigma\in D_{\frak p}\,:\,
		x^\sigma \equiv x\,(\modulo\,\frak p)\,\forall x\in\clos{\bb Z}
		\big\}
	\text{ ,}
\end{align*}
%
respectivamente, el \emph{grupo de descomposici\'{o}n de $\frak p$} y el
\emph{subgrupo de inercia}. Alternativamente, $\inercia[\frak p]$ se puede
describir como el n\'{u}cleo de un morfismo sobreyectivo
\begin{math}
	\descomp[\frak p]\rightarrow\galois{\clos{\bb F_p}/\bb F_p}
\end{math}. Sea $\sigma_p:\,x\mapsto x^p$ y se
$\frob[\frak p]\in\descomp[\frak p]$ cualquier preimagen de $\sigma_p$ por este
morfismo sobre.

Una representaci\'{o}n $\rho:\,G\rightarrow\GL[d](L)$ (morfismo continuo,
$d\geq 1$, $L/\bb Q_l$ extensi\'{o}n finita) se dice \emph{no ramificada en %
$p$}, si $\inercia[\frak p]\subset\ker(\rho)$, cualquiera sea $\frak p$ arriba
de $p$. Entonces, usando esta terminolog\'{\i}a, el criterio de
N\'{e}ron-Ogg-Shafarevich, Teorema~\ref{thm:NOS}, dice que

\begin{teoNOS}\label{thm:NOSbis}
	la rep. $\rho_{E,l}$ es no ramificada en todo primo $p$
	que no divide a $l\cdot N$.
\end{teoNOS}
