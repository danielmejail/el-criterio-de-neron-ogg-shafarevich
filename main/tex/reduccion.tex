Si $E/\cal{K}$ viene dada por una ecuaci\'{o}n minimal y $P\in\bb{P}^{2}(K)$,
se pueden hallar coordenadas homog\'{e}neas $P=[x_{0}:x_{1}:x_{2}]$, con
$x_{i}\in\cal{R}$ y, al menos, una en $\cal{R}^{\times}$. Se definen, entonces,
los conjuntos
\begin{align*}
	E_{0}(\cal{K}) & \,:=\,\big\{P\in E(\cal{K})\,:\,
		\reducido{P}\in\redns{E}(k)\big\} \quad\text{y} \\
	E_{1}(\cal{K}) & \,:=\,\big\{P\in E(\cal{K})\,:\,
		\reducido{P}=O\big\}
	\text{ .}
\end{align*}
%
Estos conjuntos son, en realidad, grupos y existe una sucesi\'{o}n exacta corta
\begin{center}
	\begin{tikzcd}
		0\arrow[r] & E_{1}(\cal{K})\arrow[r] &
			E_{0}(\cal{K})\arrow[r] & \redns{E}(k)\arrow[r] & 0
	\end{tikzcd}
\end{center}
Se hab\'{\i}a visto en clase para $\cal{K}=\bb{Q}_{p}$, pero los argumentos
valen en general.

Otra cosa que se hab\'{\i}a visto en clase fue que, para $n\geq 1$ el conjunto
\begin{align*}
	E_{n}(\cal{K}) & \,:=\,\big\{P\in E_{1}(\cal{K})\,:\,
		v(x(P))/v(y(P))\geq n\big\}\,\cup\,\{O\}
\end{align*}
%
es un subgrupo de $E_{1}(\cal{K})$ y se cumple:
\begin{itemize}
	\item[(i)]
		\begin{math}
			E_{n}(\cal{K})/E_{n+1}(\cal{K})\simeq (k,+)
		\end{math}~y
	\item[(ii)]
		\begin{math}
			\bigcap_{n\geq 1}\,E_{n}(\cal{K})=\{O\}
		\end{math}~.
\end{itemize}
%
El iso viene dado por $E_{u}(\cal{K})\xrightarrow{\phi_{u}}k$, donde
\begin{align*}
	&
	\begin{cases}
		\phi_{u}(x,y) \,=\,\pi^{-u}x/y \\
		\phi_{u}(O) \,=\,0
	\end{cases}
\end{align*}
%

\begin{propoReduccionYTorsion}\label{propo:reduccionytorsion}
	Sea $m\in\bb{Z}$ coprimo con $\char k$. Entonces
	\begin{itemize}
		\item[(i)] $E_{1}(\cal{K})$ no tiene puntos de $m$-torsi\'{o}n
			y
		\item[(ii)] si $\reducido{E}$ es no singular,
			$E(\cal{K})[m]\hookrightarrow\reducido{E}(k)$.
	\end{itemize}
	%
\end{propoReduccionYTorsion}

\begin{proof}
	Sea $P\in E(\cal{K})[m]$ y supongamos que $P\in E_{n}(\cal{K})$.
	Entonces $[m]\cdot P=O$ y, en particular,
	\begin{align*}
		m\,\phi_{n}(P) & \,=\,\phi_{n}([m]\cdot P)\,=\,0
		\text{ .}
	\end{align*}
	%
	Como $\char k\nmid m$, $m\in k^{\times}$ y $\phi_{n}(P)=0$. Esto quiere
	decir que $P\in E_{n+1}(\cal{K})$. En particular, por inducci\'{o}n,
	$E_{1}(\cal{K})$ no tiene puntos de $m$-torsi\'{o}n. Para la segunda
	parte, el n\'{u}cleo de la reduci\'{o}n es $E_{1}(\cal{K})$, pero
	$E(\cal{K})[m]\cap E_{1}(\cal{K})=\{O\}$.
\end{proof}

\begin{defReduccion}\label{def:reduccion}
	Sea $E/\cal{K}$ una curva el\'{\i}ptica dada por ecuaci\'{o}n minimal.
	Se dice que la curva (i) \emph{tiene buena reducci\'{o}n}, si
	$\reducido E(k)$ es no singular; (ii) \emph{tiene reducci\'{o}n %
	multiplicativa}, si la curva reducida es \emph{nodal}; (iii)
	\emph{tiene reducci\'{o}n aditiva}, si la curva reducida es
	\emph{cuspidal}. Cuando la reducci\'{o}n es multiplicativa, se dice,
	adem\'{a}s, que es (a) \emph{split}, si las pendientes de las
	tangentes a $\reducido E(k)$ en el nodo est\'{a}n definidas sobre $k$;
	(b) \emph{non-split}, en caso contrario.
\end{defReduccion}

\begin{obsReduccion}\label{obs:reduccion}
	Una curva dada por ecuaci\'{o}n de Weierstrass es no singular, si y
	s\'{o}lo si su discriminante es no nulo. En caso contrario, tiene un
	nodo, si ($\Delta=0$ y) $c_{4}\not=0$, y el punto singular es una
	c\'{u}spide, si ($\Delta=0$ y) $c_{4}=0$.

	La curva $E/\cal{K}$ tiene vuena reducci\'{o}n, si y s\'{o}lo si
	$v(\Delta)>0$ (equivalentemente,
	$\Delta\in\cal{R}\setmin\cal{M}=\cal{R}^{\times}$).
	Si $\Delta\equiv 0\,(\modulo\,\cal{M})$, entonces la reducci\'{o}n es
	multiplicativa, si $v(c_{4})=0$ y es aditiva, si $v(c_{4})>0$.
\end{obsReduccion}

\begin{defReduccion}\label{def:reduccionpotencialmentebuena}
	Se dice que $E/\cal{K}$ tiene \emph{mala reducci\'{o}n potencialmente %
	buena}, si, vista como una curva sobre $\cal{K}'/\cal{K}$, alguna
	extensi\'{o}n finita, tiene buena reducci\'{o}n.
\end{defReduccion}

\begin{propoReduccionBuenaYMala}\label{propo:reduccionbuenaymala}
	Sea $E/\cal{K}$ una curva el\'{\i}ptica y sea $\cal{K}'/\cal{K}$ una
	extensi\'{o}n de cuerpos.
	\begin{itemize}
		\item[(i)] Si la extensi\'{o}n es no ramificada, es decir,
			$[\cal{K}':\cal{K}]=[k':k]$, o, equivalentemente,
			$\cal{R}'\pi=\cal{M}'$, entonces el tipo de
			reducci\'{o}n de $E$ en tanto curva sobre $\cal{K}'$ es
			el mismo que en tanto curva sobre $\cal{K}$;
		\item[(ii)] si $\cal{K}'/\cal{K}$ es finita y la reducci\'{o}n
			de $E$ sobre $\cal{K}$ es buena, o bien mala y
			multiplicativa, entonces la reducci\'{o}n sobre
			$\cal{K}'$ es buena, o, respectivamente, mala y
			multiplicativa;
		\item[(iii)] en cualquier caso, existe $\cal{K}'/\cal{K}$
			finita tal que $E/\cal{K}'$ tiene reducci\'{o}n buena o
			bien split multiplicativa;
		\item[(iv)] la reducci\'{o}n es potencialmete buena, si y
			s\'{o}lo si $j(E)\in\cal{R}$.
	\end{itemize}
	%
\end{propoReduccionBuenaYMala}

\begin{proof}
	Los items (i) y (ii) se desprenden de la minimalidad de la ecuaci\'{o}n
	que define a $E$, de que $v'|_{\cal{K}}=v$, si $\cal{K}'/\cal{K}$ es no
	ramificada, y de que $v'|_{\cal{K}}$ es un m\'{u}ltiplo no nulo de $v$
	en el caso finito, haciendo cambios de variable que preserven la forma
	de Weierstrass. Para ver (iii), asumiendo caracter\'{\i}stica distinta
	de $2$, en cierta extensi\'{o}n finita $\cal{K}'/\cal{K}$, la curva
	$E/\cal{K}'$ se puede expresar en forma de Legendre:
	\begin{align*}
		E & \,:\, y^2 \,=\,x\,(x-1)\,(x-\lambda)
		\text{ ,}
	\end{align*}
	%
	con $\lambda\not=0,1$. Entonces
	\begin{align*}
		c_{4}\,=\,16\,(\lambda^{2}-\lambda+1) & \quad\text{y}\quad
		\Delta \,=\,16\,\lambda^{2}\,(\lambda-1)^{2}
		\text{ .}
	\end{align*}
	%
	\begin{itemize}
		\item[(i)] Si $\lambda\in\cal{R}'$ y $\lambda\not=0,1$ en $k'$,
			entonces $\Delta\not\equiv0\,(\modulo\,\cal{M}')$ y
			$\Delta\in{\cal{R}'}^{\times}$;
		\item[(ii)] si $\lambda =0$ o $\lambda=1$ en $k'$,
			$\Delta\equiv0\,(\modulo\,\cal{M})$, pero
			$c_4\in{\cal{R}'}^{\times}$;
		\item[(iii)] si $\lambda\not\in {\cal{R}'}^{\times}$, pasa a
			ser una unidad en $\cal{R}'$, si se lo multiplica por
			alguna potencia positiva del uniformizadr $\pi'$.
	\end{itemize}
	%
	Se hace el cambio $x=x'{\pi'}^{-r}$, $y=y'{\pi'}^{-3r/2}$, donde
	$r$ es tal que $\lambda{\pi'}^{r}\in{\cal{R}'}^{\times}$. Reemplazando,
	posiblemente, $\cal{K}'$ por una extensi\'{o}n cuadr\'{a}tica,
	\begin{align*}
		{y'}^{2}{\pi'}^{-3r} & \,=\,x'{\pi'}^{-r}\,
			(x'{\pi'}^{-r}-1)\,(x'{\pi'}^{-r}-\lambda)
			\quad\text{o bien,} \\
		{y'}^{2} & \,=\,x'\,(x'-{\pi'}^{r})\,(x'-\lambda{\pi'}^{r})
		\text{ .}
	\end{align*}
	%
	Si $\Delta'=u^{-12}\Delta$ y $c_4'=u^{-4}c_4$ son los valores asociados
	a esta ecuaci\'{o}n, entonces $\Delta'\in\cal{M}'$ y
	$c_4'\in{\cal{R}'}^{\times}$. En (i) la reducci\'{o}n es buena, en
	(ii) es multiplicativa y en (iii) es multiplicativa en una
	extensi\'{o}n, a lo sumo, cuadr\'{a}tica de $\cal{K}'$. Por otro lado,
	reducci\'{o}n non-split pasa a ser split en alguna extensi\'{o}n
	cuadr\'{a}tica.

	Para (d), asumiendo de nuevo $\char{k}\not=2$ y llevando la curva a
	forma de Legendre, se puede verificar que
	\begin{align*}
		256\,(1-\lambda\,(1-\lambda))^{3}-
			j\,\lambda^{2}\,(1-\lambda)^{2} & \,=\,0
		\text{ .}
	\end{align*}
	%
	En particular, $v(\lambda\,(1-\lambda))\geq0$. Si $v(\lambda)<0$,
	entonces
	\begin{math}
		v(\lambda\,(1-\lambda))=v(\lambda)+v(1-\lambda)=
			v(\lambda)+v(\lambda)<0
	\end{math}~, lo que es absurdo. Entonces
	\begin{align*}
		v(\lambda) & \,\geq\,0
		\text{ .}
	\end{align*}
	%
	Como $256$ es una unidad en $\cal{R}$, reduciendo, tiene que ser
	$\lambda\not\equiv0,1\,(\modulo\,\cal{M})$, y la reducci\'{o}n es
	buena.

	Si, rec\'{\i}procamente, la reducci\'{o}n es potencialmente buena, y
	buena sobre una extensi\'{o}n finita $\cal{K}'/\cal{K}$, denotando con
	$\Delta'$ el discriminante minimal de $E/\cal{K}'$ y $c_4'$ el otro
	valor asociado a la ecuaci\'{o}n minimal, se deduce que
	\begin{align*}
		j(E/\cal{K}') & \,=\,{c_4'}^{3}/\Delta'
		\text{ .}
	\end{align*}
	%
	Como $c_4'\in\cal{R}'$ y la reducci\'{o}n es buena sobre $\cal{K}'$, el
	discriminante minimal es una unidad y $j(E/\cal{K}')\in\cal{R}'$. Como
	$E$ est\'{a} definida sobre $\cal{K}$, vale que
	$j\in\cal{R}'\cap\cal{K}=\cal{R}$.
\end{proof}
