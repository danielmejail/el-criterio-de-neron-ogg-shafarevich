A modo de comentario, este criterio, junto con otros resultados no del todo
triviales, permite deducir el siguiente teorema.

\begin{teoIsogenia}\label{thm:isogenia}
	Sean $E,E'$ curvas el\'{\i}pticas sobre $\bb Q$ (o sobre un cuerpo de
	n\'{u}meros arbitrario) y sean $V_l(E),V_l(E')$ las representaciones
	asociadas v\'{\i}a el m\'{o}dulo de Tate. Si los m\'{o}dulos --es
	decir, las representaciones-- son isomorfos y $j(E)$ no es un entero,
	entonces las curvas $E$ y $E'$ son is\'{o}genas.
\end{teoIsogenia}

Hay, tambi\'{e}n, otros resultados simp\'{a}ticos que relacionan buena
reducci\'{o}n con isogen\'{\i}a:

\begin{teoBuenaRedFueraDeS}[Shafarevich]\label{thm:buenaredfuerades}
	Sea $S\subset\lugares{\cal K}$, $\#S<\infty$, entonces el conjunto de
	clases de $\cal K$-isomorfismo de curvas el\'{\i}pticas con buena
	reducci\'{o}n fuera de $S$ es finito.
\end{teoBuenaRedFueraDeS}

\begin{teoIsogeniaImplicaIsomorfismo}\label{thm:isogeniaimplicaisomorfismo}
	Dos curvas is\'{o}genas dan representaciones isomorfas.
\end{teoIsogeniaImplicaIsomorfismo}

En particular, esto implica que, si $E$ y $E'$ son $\cal K$-is\'{o}genas, buena
reducci\'{o}n de una en un lugar $v$ es acompa\~{n}ada de buena reducci\'{o}n
de la otra en el mismo lugar.

\begin{teoFinitasIsogenas}\label{thm:finitasisogenas}
	Sea $E/\cal K$ una curva el\'{\i}ptica. Salvo isomorfismo, hay una
	cantidad finita de curvas $\cal K$-is\'{o}genas a $E$.
\end{teoFinitasIsogenas}

En otras palabras, cada clase de $\cal K$-isogen\'{\i}a est\'{a} compuesta por
una cantidad finita de clases de isomorfismo.
